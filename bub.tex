%% bare_jrnl.tex
%% V1.3
%% 2007/01/11
%% by Michael Shell
%% see http://www.michaelshell.org/
%% for current contact information.
%%
%% This is a skeleton file demonstrating the use of IEEEtran.cls
%% (requires IEEEtran.cls version 1.7 or later) with an IEEE journal paper.
%%
%% Support sites:
%% http://www.michaelshell.org/tex/ieeetran/
%% http://www.ctan.org/tex-archive/macros/latex/contrib/IEEEtran/
%% and
%% http://www.ieee.org/



% *** Authors should verify (and, if needed, correct) their LaTeX system  ***
% *** with the testflow diagnostic prior to trusting their LaTeX platform ***
% *** with production work. IEEE's font choices can trigger bugs that do  ***
% *** not appear when using other class files.                            ***
% The testflow support page is at:
% http://www.michaelshell.org/tex/testflow/


%%*************************************************************************
%% Legal Notice:
%% This code is offered as-is without any warranty either expressed or
%% implied; without even the implied warranty of MERCHANTABILITY or
%% FITNESS FOR A PARTICULAR PURPOSE!
%% User assumes all risk.
%% In no event shall IEEE or any contributor to this code be liable for
%% any damages or losses, including, but not limited to, incidental,
%% consequential, or any other damages, resulting from the use or misuse
%% of any information contained here.
%%
%% All comments are the opinions of their respective authors and are not
%% necessarily endorsed by the IEEE.
%%
%% This work is distributed under the LaTeX Project Public License (LPPL)
%% ( http://www.latex-project.org/ ) version 1.3, and may be freely used,
%% distributed and modified. A copy of the LPPL, version 1.3, is included
%% in the base LaTeX documentation of all distributions of LaTeX released
%% 2003/12/01 or later.
%% Retain all contribution notices and credits.
%% ** Modified files should be clearly indicated as such, including  **
%% ** renaming them and changing author support contact information. **
%%
%% File list of work: IEEEtran.cls, IEEEtran_HOWTO.pdf, bare_adv.tex,
%%                    bare_conf.tex, bare_jrnl.tex, bare_jrnl_compsoc.tex
%%*************************************************************************

% Note that the a4paper option is mainly intended so that authors in
% countries using A4 can easily print to A4 and see how their papers will
% look in print - the typesetting of the document will not typically be
% affected with changes in paper size (but the bottom and side margins will).
% Use the testflow package mentioned above to verify correct handling of
% both paper sizes by the user's LaTeX system.
%
% Also note that the "draftcls" or "draftclsnofoot", not "draft", option
% should be used if it is desired that the figures are to be displayed in
% draft mode.
%
\documentclass[journal]{IEEEtran}
%
% If IEEEtran.cls has not been installed into the LaTeX system files,
% manually specify the path to it like:
% \documentclass[journal]{../sty/IEEEtran}





% Some very useful LaTeX packages include:
% (uncomment the ones you want to load)


% *** MISC UTILITY PACKAGES ***
%
%\usepackage{ifpdf}
% Heiko Oberdiek's ifpdf.sty is very useful if you need conditional
% compilation based on whether the output is pdf or dvi.
% usage:
% \ifpdf
%   % pdf code
% \else
%   % dvi code
% \fi
% The latest version of ifpdf.sty can be obtained from:
% http://www.ctan.org/tex-archive/macros/latex/contrib/oberdiek/
% Also, note that IEEEtran.cls V1.7 and later provides a builtin
% \ifCLASSINFOpdf conditional that works the same way.
% When switching from latex to pdflatex and vice-versa, the compiler may
% have to be run twice to clear warning/error messages.






% *** CITATION PACKAGES ***
%
\usepackage{cite}
% cite.sty was written by Donald Arseneau
% V1.6 and later of IEEEtran pre-defines the format of the cite.sty package
% \cite{} output to follow that of IEEE. Loading the cite package will
% result in citation numbers being automatically sorted and properly
% "compressed/ranged". e.g., [1], [9], [2], [7], [5], [6] without using
% cite.sty will become [1], [2], [5]--[7], [9] using cite.sty. cite.sty's
% \cite will automatically add leading space, if needed. Use cite.sty's
% noadjust option (cite.sty V3.8 and later) if you want to turn this off.
% cite.sty is already installed on most LaTeX systems. Be sure and use
% version 4.0 (2003-05-27) and later if using hyperref.sty. cite.sty does
% not currently provide for hyperlinked citations.
% The latest version can be obtained at:
% http://www.ctan.org/tex-archive/macros/latex/contrib/cite/
% The documentation is contained in the cite.sty file itself.






% *** GRAPHICS RELATED PACKAGES ***
%
\ifCLASSINFOpdf
  % \usepackage[pdftex]{graphicx}
  % declare the path(s) where your graphic files are
  % \graphicspath{{../pdf/}{../jpeg/}}
  % and their extensions so you won't have to specify these with
  % every instance of \includegraphics
  % \DeclareGraphicsExtensions{.pdf,.jpeg,.png}
\else
  % or other class option (dvipsone, dvipdf, if not using dvips). graphicx
  % will default to the driver specified in the system graphics.cfg if no
  % driver is specified.
   \usepackage[dvips]{graphicx}
  % declare the path(s) where your graphic files are
  % \graphicspath{{../eps/}}
  % and their extensions so you won't have to specify these with
  % every instance of \includegraphics
  % \DeclareGraphicsExtensions{.eps}
\fi
% graphicx was written by David Carlisle and Sebastian Rahtz. It is
% required if you want graphics, photos, etc. graphicx.sty is already
% installed on most LaTeX systems. The latest version and documentation can
% be obtained at:
% http://www.ctan.org/tex-archive/macros/latex/required/graphics/
% Another good source of documentation is "Using Imported Graphics in
% LaTeX2e" by Keith Reckdahl which can be found as epslatex.ps or
% epslatex.pdf at: http://www.ctan.org/tex-archive/info/
%
% latex, and pdflatex in dvi mode, support graphics in encapsulated
% postscript (.eps) format. pdflatex in pdf mode supports graphics
% in .pdf, .jpeg, .png and .mps (metapost) formats. Users should ensure
% that all non-photo figures use a vector format (.eps, .pdf, .mps) and
% not a bitmapped formats (.jpeg, .png). IEEE frowns on bitmapped formats
% which can result in "jaggedy"/blurry rendering of lines and letters as
% well as large increases in file sizes.
%
% You can find documentation about the pdfTeX application at:
% http://www.tug.org/applications/pdftex





% *** MATH PACKAGES ***
%
\usepackage[cmex10]{amsmath}
\usepackage{amsfonts}
% A popular package from the American Mathematical Society that provides
% many useful and powerful commands for dealing with mathematics. If using
% it, be sure to load this package with the cmex10 option to ensure that
% only type 1 fonts will utilized at all point sizes. Without this option,
% it is possible that some math symbols, particularly those within
% footnotes, will be rendered in bitmap form which will result in a
% document that can not be IEEE Xplore compliant!
%
% Also, note that the amsmath package sets \interdisplaylinepenalty to 10000
% thus preventing page breaks from occurring within multiline equations. Use:
%\interdisplaylinepenalty=2500
% after loading amsmath to restore such page breaks as IEEEtran.cls normally
% does. amsmath.sty is already installed on most LaTeX systems. The latest
% version and documentation can be obtained at:
% http://www.ctan.org/tex-archive/macros/latex/required/amslatex/math/





% *** SPECIALIZED LIST PACKAGES ***
%
%\usepackage{algorithmic}
% algorithmic.sty was written by Peter Williams and Rogerio Brito.
% This package provides an algorithmic environment fo describing algorithms.
% You can use the algorithmic environment in-text or within a figure
% environment to provide for a floating algorithm. Do NOT use the algorithm
% floating environment provided by algorithm.sty (by the same authors) or
% algorithm2e.sty (by Christophe Fiorio) as IEEE does not use dedicated
% algorithm float types and packages that provide these will not provide
% correct IEEE style captions. The latest version and documentation of
% algorithmic.sty can be obtained at:
% http://www.ctan.org/tex-archive/macros/latex/contrib/algorithms/
% There is also a support site at:
% http://algorithms.berlios.de/index.html
% Also of interest may be the (relatively newer and more customizable)
% algorithmicx.sty package by Szasz Janos:
% http://www.ctan.org/tex-archive/macros/latex/contrib/algorithmicx/




% *** ALIGNMENT PACKAGES ***
%
%\usepackage{array}
% Frank Mittelbach's and David Carlisle's array.sty patches and improves
% the standard LaTeX2e array and tabular environments to provide better
% appearance and additional user controls. As the default LaTeX2e table
% generation code is lacking to the point of almost being broken with
% respect to the quality of the end results, all users are strongly
% advised to use an enhanced (at the very least that provided by array.sty)
% set of table tools. array.sty is already installed on most systems. The
% latest version and documentation can be obtained at:
% http://www.ctan.org/tex-archive/macros/latex/required/tools/


%\usepackage{mdwmath}
%\usepackage{mdwtab}
% Also highly recommended is Mark Wooding's extremely powerful MDW tools,
% especially mdwmath.sty and mdwtab.sty which are used to format equations
% and tables, respectively. The MDWtools set is already installed on most
% LaTeX systems. The lastest version and documentation is available at:
% http://www.ctan.org/tex-archive/macros/latex/contrib/mdwtools/


% IEEEtran contains the IEEEeqnarray family of commands that can be used to
% generate multiline equations as well as matrices, tables, etc., of high
% quality.


%\usepackage{eqparbox}
% Also of notable interest is Scott Pakin's eqparbox package for creating
% (automatically sized) equal width boxes - aka "natural width parboxes".
% Available at:
% http://www.ctan.org/tex-archive/macros/latex/contrib/eqparbox/





% *** SUBFIGURE PACKAGES ***
\usepackage[tight,footnotesize]{subfigure}
% subfigure.sty was written by Steven Douglas Cochran. This package makes it
% easy to put subfigures in your figures. e.g., "Figure 1a and 1b". For IEEE
% work, it is a good idea to load it with the tight package option to reduce
% the amount of white space around the subfigures. subfigure.sty is already
% installed on most LaTeX systems. The latest version and documentation can
% be obtained at:
% http://www.ctan.org/tex-archive/obsolete/macros/latex/contrib/subfigure/
% subfigure.sty has been superceeded by subfig.sty.



%\usepackage[caption=false]{caption}
%\usepackage[font=footnotesize,caption=false]{subfig}
% subfig.sty, also written by Steven Douglas Cochran, is the modern
% replacement for subfigure.sty. However, subfig.sty requires and
% automatically loads Axel Sommerfeldt's caption.sty which will override
% IEEEtran.cls handling of captions and this will result in nonIEEE style
% figure/table captions. To prevent this problem, be sure and preload
% caption.sty with its "caption=false" package option. This is will preserve
% IEEEtran.cls handing of captions. Version 1.3 (2005/06/28) and later
% (recommended due to many improvements over 1.2) of subfig.sty supports
% the caption=false option directly:
%\usepackage[caption=false,font=footnotesize]{subfig}
%
% The latest version and documentation can be obtained at:
% http://www.ctan.org/tex-archive/macros/latex/contrib/subfig/
% The latest version and documentation of caption.sty can be obtained at:
% http://www.ctan.org/tex-archive/macros/latex/contrib/caption/




% *** FLOAT PACKAGES ***
%
%\usepackage{fixltx2e}
% fixltx2e, the successor to the earlier fix2col.sty, was written by
% Frank Mittelbach and David Carlisle. This package corrects a few problems
% in the LaTeX2e kernel, the most notable of which is that in current
% LaTeX2e releases, the ordering of single and double column floats is not
% guaranteed to be preserved. Thus, an unpatched LaTeX2e can allow a
% single column figure to be placed prior to an earlier double column
% figure. The latest version and documentation can be found at:
% http://www.ctan.org/tex-archive/macros/latex/base/



%\usepackage{stfloats}
% stfloats.sty was written by Sigitas Tolusis. This package gives LaTeX2e
% the ability to do double column floats at the bottom of the page as well
% as the top. (e.g., "\begin{figure*}[!b]" is not normally possible in
% LaTeX2e). It also provides a command:
%\fnbelowfloat
% to enable the placement of footnotes below bottom floats (the standard
% LaTeX2e kernel puts them above bottom floats). This is an invasive package
% which rewrites many portions of the LaTeX2e float routines. It may not work
% with other packages that modify the LaTeX2e float routines. The latest
% version and documentation can be obtained at:
% http://www.ctan.org/tex-archive/macros/latex/contrib/sttools/
% Documentation is contained in the stfloats.sty comments as well as in the
% presfull.pdf file. Do not use the stfloats baselinefloat ability as IEEE
% does not allow \baselineskip to stretch. Authors submitting work to the
% IEEE should note that IEEE rarely uses double column equations and
% that authors should try to avoid such use. Do not be tempted to use the
% cuted.sty or midfloat.sty packages (also by Sigitas Tolusis) as IEEE does
% not format its papers in such ways.


%\ifCLASSOPTIONcaptionsoff
%  \usepackage[nomarkers]{endfloat}
% \let\MYoriglatexcaption\caption
% \renewcommand{\caption}[2][\relax]{\MYoriglatexcaption[#2]{#2}}
%\fi
% endfloat.sty was written by James Darrell McCauley and Jeff Goldberg.
% This package may be useful when used in conjunction with IEEEtran.cls'
% captionsoff option. Some IEEE journals/societies require that submissions
% have lists of figures/tables at the end of the paper and that
% figures/tables without any captions are placed on a page by themselves at
% the end of the document. If needed, the draftcls IEEEtran class option or
% \CLASSINPUTbaselinestretch interface can be used to increase the line
% spacing as well. Be sure and use the nomarkers option of endfloat to
% prevent endfloat from "marking" where the figures would have been placed
% in the text. The two hack lines of code above are a slight modification of
% that suggested by in the endfloat docs (section 8.3.1) to ensure that
% the full captions always appear in the list of figures/tables - even if
% the user used the short optional argument of \caption[]{}.
% IEEE papers do not typically make use of \caption[]'s optional argument,
% so this should not be an issue. A similar trick can be used to disable
% captions of packages such as subfig.sty that lack options to turn off
% the subcaptions:
% For subfig.sty:
% \let\MYorigsubfloat\subfloat
% \renewcommand{\subfloat}[2][\relax]{\MYorigsubfloat[]{#2}}
% For subfigure.sty:
% \let\MYorigsubfigure\subfigure
% \renewcommand{\subfigure}[2][\relax]{\MYorigsubfigure[]{#2}}
% However, the above trick will not work if both optional arguments of
% the \subfloat/subfig command are used. Furthermore, there needs to be a
% description of each subfigure *somewhere* and endfloat does not add
% subfigure captions to its list of figures. Thus, the best approach is to
% avoid the use of subfigure captions (many IEEE journals avoid them anyway)
% and instead reference/explain all the subfigures within the main caption.
% The latest version of endfloat.sty and its documentation can obtained at:
% http://www.ctan.org/tex-archive/macros/latex/contrib/endfloat/
%
% The IEEEtran \ifCLASSOPTIONcaptionsoff conditional can also be used
% later in the document, say, to conditionally put the References on a
% page by themselves.





% *** PDF, URL AND HYPERLINK PACKAGES ***
%
%\usepackage{url}
% url.sty was written by Donald Arseneau. It provides better support for
% handling and breaking URLs. url.sty is already installed on most LaTeX
% systems. The latest version can be obtained at:
% http://www.ctan.org/tex-archive/macros/latex/contrib/misc/
% Read the url.sty source comments for usage information. Basically,
% \url{my_url_here}.





% *** Do not adjust lengths that control margins, column widths, etc. ***
% *** Do not use packages that alter fonts (such as pslatex).         ***
% There should be no need to do such things with IEEEtran.cls V1.6 and later.
% (Unless specifically asked to do so by the journal or conference you plan
% to submit to, of course. )


% correct bad hyphenation here
\hyphenation{op-tical net-works semi-conduc-tor}


\begin{document}
%
% paper title
% can use linebreaks \\ within to get better formatting as desired
\title{Bottom-up Broadband: Free Software Philosophy Applied To Networking Initiatives}
%
%
% author names and IEEE memberships
% note positions of commas and nonbreaking spaces ( ~ ) LaTeX will not break
% a structure at a ~ so this keeps an author's name from being broken across
% two lines.
% use \thanks{} to gain access to the first footnote area
% a separate \thanks must be used for each paragraph as LaTeX2e's \thanks
% was not built to handle multiple paragraphs
%

\author{
	Name~Surname,~%\IEEEmembership{Member,~IEEE,}
    Name~Surname,~%\IEEEmembership{Member,~IEEE,}
    Name~Surname,~%\IEEEmembership{Member,~IEEE,}
    Name~Surname,~%\IEEEmembership{Member,~IEEE,}
    and~Name~Surname%~\IEEEmembership{Life~Fellow,~IEEE}% <-this % stops a space
\thanks{The authors are with bub univeristy}
}


% note the % following the last \IEEEmembership and also \thanks -
% these prevent an unwanted space from occurring between the last author name
% and the end of the author line. i.e., if you had this:
%
% \author{....lastname \thanks{...} \thanks{...} }
%                     ^------------^------------^----Do not want these spaces!
%
% a space would be appended to the last name and could cause every name on that
% line to be shifted left slightly. This is one of those "LaTeX things". For
% instance, "\textbf{A} \textbf{B}" will typeset as "A B" not "AB". To get
% "AB" then you have to do: "\textbf{A}\textbf{B}"
% \thanks is no different in this regard, so shield the last } of each \thanks
% that ends a line with a % and do not let a space in before the next \thanks.
% Spaces after \IEEEmembership other than the last one are OK (and needed) as
% you are supposed to have spaces between the names. For what it is worth,
% this is a minor point as most people would not even notice if the said evil
% space somehow managed to creep in.



% The paper headers
%\markboth{Journal of \LaTeX\ Class Files,~Vol.~6, No.~1, January~2007}%
%{Shell \MakeLowercase{\textit{et al.}}: Bare Demo of IEEEtran.cls for Journals}
% The only time the second header will appear is for the odd numbered pages
% after the title page when using the twoside option.
%
% *** Note that you probably will NOT want to include the author's ***
% *** name in the headers of peer review papers.                   ***
% You can use \ifCLASSOPTIONpeerreview for conditional compilation here if
% you desire.




% If you want to put a publisher's ID mark on the page you can do it like
% this:
%\IEEEpubid{0000--0000/00\$00.00~\copyright~2007 IEEE}
% Remember, if you use this you must call \IEEEpubidadjcol in the second
% column for its text to clear the IEEEpubid mark.



% use for special paper notices
%\IEEEspecialpapernotice{(Invited Paper)}




% make the title area
\maketitle


\begin{abstract}
%\boldmath

This paper discusses open networks.
The free software and open hardware movements are relatively well established and known.
Contrastingly, there is little discussion on open network initiatives.
Software, hardware and networks are closely knitted together and therefore it makes sense to explore open networks by establishing parallelisms with free software.

The first part of this paper presents a classification of open networks, according to the degree of openness.
Then we study the driving principles behind open networks to see that these principles are not that different from those found in free software development.
The community-centered approach that has allowed the growth and the success of free software may as well represent the key strength of open networks.
We point out the advantages of the peer-to-peer production model found in the communities that work on free networks.
Finally, we introduce the Bottom-up Broadband project that has the goal to study and promote open networks.


\end{abstract}
% IEEEtran.cls defaults to using nonbold math in the Abstract.
% This preserves the distinction between vectors and scalars. However,
% if the journal you are submitting to favors bold math in the abstract,
% then you can use LaTeX's standard command \boldmath at the very start
% of the abstract to achieve this. Many IEEE journals frown on math
% in the abstract anyway.

% Note that keywords are not normally used for peerreview papers.
%\begin{IEEEkeywords}
% media access control, WLAN, collision-free schedule.
%Slotted Aloha, game theory, contention control, media access control.
%\end{IEEEkeywords}






% For peer review papers, you can put extra information on the cover
% page as needed:
% \ifCLASSOPTIONpeerreview
% \begin{center} \bfseries EDICS Category: 3-BBND \end{center}
% \fi
%
% For peerreview papers, this IEEEtran command inserts a page break and
% creates the second title. It will be ignored for other modes.
\IEEEpeerreviewmaketitle

\section{Disclaimer}

This is just a draft/brainstorm document.
There is people that pointed out that there are aspects that are openly wrong.

Comments, suggestions, criticisms, contributions and pull requests are welcome.

\section{Introduction}
% The very first letter is a 2 line initial drop letter followed
% by the rest of the first word in caps.
%
% form to use if the first word consists of a single letter:
% \IEEEPARstart{A}{demo} file is ....
%
% form to use if you need the single drop letter followed by
% normal text (unknown if ever used by IEEE):
% \IEEEPARstart{A}{}demo file is ....
%
% Some journals put the first two words in caps:
% \IEEEPARstart{T}{his demo} file is ....
%
% Here we have the typical use of a "T" for an initial drop letter
% and "HIS" in caps to complete the first word.
\IEEEPARstart{O}{pen} source is a way of creating and sharing software that has several advantages and areas of applicability.
By publishing the source code instead of keeping it secret, a production model that relies on cooperation instead of competence is encouraged.
Free software projects are often in hands of communities instead of corporations, and the skilled users have an opportunity to adapt it to their needs and shape the evolution of the software.

The open source model for software development is currently well accepted and we find free software running almost everywhere: personal computers, mobile phones, tablets, server farms and embedded devices.

In this paper we discuss the application of open source development principles to network deployment.
There is a whole spectrum of diversity between totally closed networks and free (libre) networks.
We classify open networking initiatives in three categories and provide examples for each of them.

We also discuss how the principles of community development have some advantages that make it possible to build better networks, or simply build networks that are not viable when using more closed approaches.

%We emphasize the benefits of open networking for research and education, as the openness naturally leads to symbiotic collaborations.

We also briefly describe the \emph{Bottom-up Broadband for Europe} initiative that has the goal of studying and promoting network deployments in which the users are active participants and not mere passive consumers.
We use the term bottom-up to emphasize the fact that are the users the ones that take the initiative.

\section{A Classification of Open Networks}
When it comes to openness, each network has its own particularities.
We propose a classification that encompasses three levels of openness: shared Internet access, open access networks and free (libre) networks.
Examples are provided for each kind of network to illustrate the concepts.

These models can be compared or contrasted with what we call the top-down monolithically integrated model.
In the monolithically integrated model, there is a telecom operator that owns the infrastructure and provides the service to a customer, which is the only one that uses it.



\subsection{Shared Internet Access}

Sharing Internet access is a first step away from vertically integrated monolithical models.
In shared Internet access, a person or entity that has Internet access in a given location sees an advantage in sharing it with someone else.
This sharing can often result in a win-win situation for all the participants.
We provide three examples of such sharing.

\emph{Eduroam} is an international WiFi roaming service for members of education institutions.
It is useful for visiting scholars and for students using libraries of different universities.
All of them can use their home university credentials to access the Internet from the premises of any other affiliated institution.
It is useful for the visitor and also for the host institution that benefits from the visit.

ProvinciaWiFi is a WiFi service available in the province of Rome and other regions in Italy that offers WiFi access in public locations.
Many commerces collaborate and share their own bandwidth with ProvinciaWifi to attract and retain customers.
In this case, the commerces acquire an access point with ProvinciWifi's open firmware called OpenWisp, and ProvinciaWifi takes care of the user authentication according Italian law.

Finally, FON is an example of a business model built on collaboration.
Collaborating members of FON install a FON access point to their Internet connection and share that connection with other users.
This becomes a FON hotspot.
The members of FON can connect to any of the millions of FON hotspots worldwide for free.
Non-members can also connect, but they have to pay.
FON has partnered with large Telecom operators, such as BT and Deutsche Telekom, that have recognized the benefits of network sharing.


\subsection{Open Access Networks}

Open Access Networks present a layered model in which different service providers can share the same infrastructure.
The participating entities agree in a set of rules that govern their collaboration.
In a basic model \cite{battiti2005wireless}, a \emph{neutral operator} takes care of the infrastructure.
Then, multiple service providers offer a multiplicity of services such as TV, telephone and Internet access over the shared open access network.
The service providers use a share of their incomes to pay to the network operator for the common infrastructure.

This model represents savings for the service providers, as they can reach a large number of customers without incurring in costly infrastructure deployment.
Furthermore, it lowers the entry barrier for new service providers and fosters competition and innovation.
The users have a larger number of options to choose from and select the service providers that better adapt to their needs \cite{domingo2011med}.
As it is easy for an user to switch from one service provider to another, the service providers have a good incentive to keep their customers happy.
Happy customers spend more money on communication services which is, in turn, beneficial for the service providers.

In this model, the final users, house owners, the City Hall or other organizations may have interest in collaborating in the extension of the access network.
A single investment in a neighborhood will provide the citizens with a diverse offering of services.
Forzati et al. report that 95\% of the municipality networks in Sweden operate according open access network models \cite{forzati2010open}.

The neutral operator is in a privileged position as it owns the network that all the others ISPs need to use.
Consequently, it is recommended that the neutral operator is a \emph{for benefit} (not for profit) organization.
The neutral operator should not be allowed to offer services to the final user, and should treat all service providers equally, following some agreed rules \cite{battiti2005wireless}.
The service providers still operate under a market competition logic.
Nevertheless, in spite of being competitors, the service providers need to collaborate to maintain the neutral operator.

\subsection{Free (Libre) Networks}

Free (libre) networks are the network equivalent of free (libre) software.
They are community-oriented and governed by rules that emphasize freedom.
\emph{guifi.net} is an example initiative of a free network and operates according to four basic pillars:
\begin{itemize}
\item Freedom to use the network, as long the other users, the contents, and the network itself are respected.
\item Freedom to learn the working details of network elements and the network as a whole. Freedom to disseminate the knowledge and the spirit of the network.
\item Freedom to offer services and contents.
\item By joining the network, the network is extended according to the previous principles.
\end{itemize}

There are other community networks operating under similar principles.
\emph{guifi.net} has the merit of having extended beyond the technical community and currently has 21,000 working nodes and is growing at a steady pace.
This growth has been enabled by the development of automated tools to configure and manage the nodes.
Another remarkable aspect is that \emph{guifi.net} has evolved since its origins to embrace new technologies, such as optical fiber and dynamic wireless mesh solutions.
Optical fiber offers gigabit per second speeds to the participants of the network and dynamic wireless mesh makes it possible to have out-of-the box quick network deployments.

Regarding funding, the wireless network has grown organically in a model in which the people that joins the network contributes with a new node.
Crowdsourcing is used to pay for more ambitious actions that benefit many participants.
Optical fiber deployment has been a new challenge in terms of funding, as the interested participants had to pay in advance to cover the cost of the deployment.
A model similar to the open access network mentioned in the previous subsection has been adopted.
Multiple service providers are allowed to operate over the common infrastructure and they have to collaborate in the deployment and maintenance of such infrastructure.
Even though the users have to pay for the initial inversion, this is compensated by the lower monthly fees and the higher connection speed, which is currently 1 Gbps.

Interestingly,  the bottom-up fiber deployment has occurred in a rural region to connect farms that otherwise could not enjoy network speeds parallel to those offered in the city.
Even though the region was right next to a fiber backbone, the rural market was not attractive for traditional top-down operators.
Top down vertically integrated operators prefer to focus their efforts in urban deployments which are more profitable.
By changing the model from top-down to bottom-up, it was possible to offer a high quality service to the users and provide business to small local ISPs.

The availability of broadband connection is a clear advantage for the development of the region and the presence of a free network in the area has positively impacted the digital inclusion indicators \cite{oliver2010wca}.

\section{Evolution and new trends in open access}

The move from closed networks to open or free networks does not happen overnight.
It is a progressive change which connects with other opening processes, such a as free software or open data.
Here we provide the view of the evolution towards free networks offered by the Free Network Foundation and also explain how open access sensor networks can be an enabler for open data.


\subsection{Steps towards a global free network}
The Free Network Foundation wiki describes that the transition towards a global free network is going to be progressive.
In a first step, neighbours will create a small network to share an Internet access.
This is already quite common, as sharing makes it possible to take advantage of statistical multiplexing.
Many users are not making an intense use the network the whole day and therefore they can save some money sharing the connexion.

In a second step, the participants that created a network to share Internet access start to offer services in the network.
Network logging and monitoring servers, web servers, proxy servers, mailing lists, internet relay chat servers, blog servers and ftp servers are natural choices.
At this point, the bottom-up broadband network is not only a content consumer.
It is also a content provider.

The third step is to interconnect the bottom-up broadband islands using tunnels.
And the fourth step is to deploy backbone links to replace the tunnels.

The final goal of the bottom-up broadband network should be to provide Internet access to anyone that is interested, to prevent that price barriers leave a fraction of the population without access to the Internet.


\subsection{Open Sensory Data}

An area that has been subject to intense research in the last years is sensor networks.
This is still an emerging technology that promises to make it possible to gather large quantities of data that then can be used to make decisions or offer services.

%We can find different approaches in the deployment of sensor networks and the management of the collected data.
%In a closed approach, an individual or organization can use a closed solution (closed hardware, closed software) to gather the data that then will be kept secret.
The principles of openness can also be applied to sensor networks.
There are initiatives, such as the Air Quality Egg or the Smart Citizen Kit, that embrace a community-driven solution in which open hardware and free software is used to create the network.
These initiatives have the complicity of the hacker and do-it-yourself communities to adjust, refine and enhance the solutions.
Importantly, the data is gathered and owned by the users that can choose to share it as open data for others to create applications and value on top of it.


\section{Peer-to-Peer Production Model}
The peer-to-peer production model \cite{bauwens2009class} relies on horizontal symmetric relations.
It is in contrast to the hierarchical boss-to-subordinate model and the consumer-producer model.
This model is well understood by the networking community as there is a large experience with peer-to-peer distributed applications in which the participating nodes take turns in assuming the roles of client and servers.
In the networked applications realm, peer-to-peer offers advantages such as scalability and resilience, and also some challenges such as bootstraping or providing quality-of-service guarantees.

Peer-to-peer is also a form of organization for the people.
There are examples of people collaborating as equals in the production of goods: the peer-to-peer production model.
Open source development or the construction of the Wikipedia involve a large number of peers contributing in the creation of value.
In fact, peer-to-peer models have shown to be more effective in particular areas than hierarchical models.

Even though all peers are in principle equal, not all the contributions necessarily are.
It is normal that there are contributions of high value while others are of dubious value.
In any case, the peer-to-peer production model is inclusive by definition and offers an opportunity to all of those acting in good faith.
Even for those that have little to contribute, the peer-to-peer model offers the possibility of learning and being part of the project.

The peer-to-peer model relies on transparency and openness, ethical principles and good practices.
These driving principles derive in what it has been termed \emph{hyperproductivity}, in the sense that they are more effective than hierarchical models.
The higher efficiency of peer production models may result in the progressive displacement and replacement of previous production models.
The changes occur progressively and gradually.
In the following we detail some specific aspects that characterize peer-to-peer models.

\subsection{Throwing more brains into the project}

The model of the \emph{bazaar}, as opposed to the \emph{cathedral} \cite{raymond1999cb}, emphasizes the communication with the users and their participation and collaboration in the development process.
Having more people thinking and working for the project improves and speeds up the development.
Moreover, by providing the users all the information about the network, it is possible for them to provide high-quality feedback and help.

Another aspect is that building a network requires a large set of skills, including purchasing, installation, configuration, programming, design, financial, legal, etc.
The person that is good at mounting a pole on top of the roof is not necessarily the one that has server administration skills.
By having a large number of people involved, all the necessary skills can be covered by one or more of the participants.

Furthermore, people have the chance of doing what they like to do and what they are good at.
The person that enjoys aligning antennas will probably be good at it and will do it with pleasure whenever it is needed.
Someone else will prefer to prepare video tutorials.
Ideally, everyone can find a way in which she can efficiently collaborate with the project.
Giving the people the opportunity to decide in which way they want to collaborate is one of the key advantages of the peer-to-peer model compared to the top-down approach.


Letting people participate in the decision-making process and perform the tasks they are interested in results in a higher motivation, that also helps to improve the efficiency of the peer-to-peer model.

Working with pleasure is not the same as working for free.
The services to the community are intertwined with monetary payments that will allow the main contributors to devote more time to the peer-to-peer project and make a living out of it.
These aspects are explored in the next subsection.

\subsection{The peer-to-peer economic ecosystem}

Bottom-up Broadband initiatives normally begin with a purpose of social service.
Quite often, the participants are interested in satisfying their own networking needs and those of their communities.
It is not uncommon that each participant buys her own hardware to be part of the network.
As the building of a data communication networks results in the simultaneous construction of a network of trust, the participants team up and collaborate using crowdfunding schemes to buy equipment that is perceived to be beneficial for all.
For the most ambitious initiatives, such as optical fiber networks, the participants need to reach an agreement and advance an important quantity of money to make the deployment possible.
Sometimes, local authorities perceive that there is a value in supporting the deployment of networks and also contribute to these initiatives.

Even though the main motivation is not to make money out of the network, an economic ecosystem is created to support it.
For instance, hardware merchants can specialize in selling the equipment that is needed for the network.
Also installers have an opportunity to work for those users that cannot install and configure the equipment by themselves.
This is specially critical in more complex deployments such as optical fiber networks.
Finally, it is also possible to create companies that sell services, such as Internet access or TV on top of the bottom-up network.
The result is that in addition to offering broadband to the users, the network also offers jobs and stimulates the local economy.

The companies around bottom-up deployments are leaded by local enterpreneurs, which also spend their money locally and in this way maintain the wealth of the community.
The small companies that make a living on the network have a good incentive to invest in such network, and therefore reinforce the Bottom-up Broadband initiative.
Nevertheless, these companies operate in the frontier between the peer-to-peer economy and the market economy and tension and disputes arise.
The companies are natural competitors but they have to invest in the same commons resource.
It can be tempting for one company to stop investing in the network in order to obtain a competitive advantage over the other.
However, this move can be perceived by the community as treachery.
In this case the community may compensate the situation by favoring the company that gives more to the network.
This presents a delicate equilibrium that not always can resolve satisfactorily.

It is often the case that the participants agree in a contract and set of rules intended to prevent any kind of abuse.
Nevertheless, for the project to progress smoothly, it is important that the participants are as interested in contributing as they are in profiting from the network.

In the ideal case, the situation is not that different from the one existing in the open source ecosystem, in which there are companies that make a living out of open source and generously donate code or employees to the community in order to strengthen the ecosystem.

In the case of open sensor networks, a possible sustainability model is the selling of open hardware.
This approach has already been successfully executed by open hardware designers and manufacturers, such as Arduino an Raspberry.
The product is open hardware and therefore can benefit from the efforts and contributions of the community.
At the same time, most people will prefer to buy the product rather than produce it themselves.
The incomes originating from selling devices can be used to maintain a team of core developers and reinvest in improving the product.


\subsection{From scarcity to abundance}

In a market economy, the resources are scarce and therefore they can be sold to obtain the maximum profit.
In an abundance economy, it is no longer possible to speculate with the goods as there are more goods than needed.
In these economies, the goal of the participants is no longer to accumulate wealth, but to build up reputation.
Reputation is obtained by giving gifts, and for this reasons this model also receives the name of gift-economy \cite{barbrook1998htg}.

There is already a gift economy for Internet content.
Photographers donate photos, programmers donate programs and bloggers donate articles.
It works because there is a wealth of information and the users cannot consume it all.
Users can receive without contributing, but still many participants decide to contribute because it \emph{feels good}, and they like to participate in this reputation system in which those that contribute more receive the recognition of the community.
With digital goods, its easy to reach gift economies because the cost of replicating and distributing the goods is very low.

For network deployment, the situation is not that obvious.
Still, it is not uncommon to find situations in which by contributing one node to the network, a participant receives much more than offers.
These situations favor the transition to a gift-economy.
The spirit of the gift economy is often present in bottom-up networks mailing lists and community gatherings, in which participants uninterestedly help each others in a continuous process of learning.

%\subsection{Advantages for research and education}

%The open models lean themselves to collaboration with research and education.
%The fact that all information and knowledge are shared makes it possible for students to understand the whole picture and know the working details.
%Moreover, there are often tasks in which the students can collaborate.
%By collaborating, the students acquire a higher quality knowledge that by simply looking, reading or listening.
%Collaboration forces them to actually think to take decisions.
%Taking decisions also means making mistakes, but the tasks handed to the students are chosen in such a ways that the mistakes are not necessarily a problem.
%Quite the opposite, they can be a valuable step of the learning process.

%As the strength of a community network resides precisely in the people, community networks place a focus on education.
%Workshops, tutorials and discussions educate and empower the users that then can contribute to the network.
%Reaching a critical mass of educated users make it possible to continuously innovate and evolve.

%Regarding research, a good knowledge of the overall system is necessary to correctly interpret the results.
%Furthermore, having access to the internals of the network makes it possible to design and run experiments that would be impossible in closed networks.

%A large fraction of the research community is already familiar with open and sharing models.
%Just as it happens with networks, sharing the scientific results and interconnecting them makes them more valuable.
%Providing all the internal details of the experiments makes them reproducible and allow other teams to build on previous work.


\section{The Bottom-up Broadband for Europe}

The \emph{Bottom-up Broadband for Europe} initiative studies and encourages bottom-up networking deployments.
By bottom-up, we mean that there is a peer-to-peer ingredient and that the users play a more active role than mere passive consumers.
The project has two differentiated goals.
In the short term, the objective is to run experiments (called pilots) to learn about bottom-up initiatives, create awareness and interconnect disperse efforts.
In a second stage of the project, the intention is to create a global community or umbrella organization that helps in promoting and coordinating the multitude of initiatives that are taking place.

\emph{Bottom-up Broadband for Europe} tries to cover a wide range of pilots in terms of technologies being used, geographical location, funding model, number of users, etc.
It also encourages the collaboration between students and experts and the preparation of profuse documentation that can help to repeat successes, avoid pitfalls, and guide policymakers.

We try to apply the same principles that make free libre software and free libre networks so successful.
We run a public mailing list and everyone is invited to contribute and collaborate.

\section{Conclusion} \label{sec:conclusion}

In this paper we have discussed open networks and how they are similar to free software.
The focus is placed on the peer-to-peer organizational models that have shown to be at the same time flexible and effective.
The clearest example of the success of this approach is the deployment of optical networks to interconnect farms in a rural area.
%In addition to the benefits of having broadband networks, these initiatives also offer advantages in terms of education, research and local economy.
Open networking is just a new way of building networks that encourages collaboration instead of competition.
\emph{Bottom-up Broadband for Europe} has been created to study, document and promote open networks with the ultimate goal of eradicating the digital divide.
We are witnesses of a transformation in which the network users take a more active role and participate and guide the creation and development of networks driven by social motivations.


% An example of a floating figure using the graphicx package.
% Note that \label must occur AFTER (or within) \caption.
% For figures, \caption should occur after the \includegraphics.
% Note that IEEEtran v1.7 and later has special internal code that
% is designed to preserve the operation of \label within \caption
% even when the captionsoff option is in effect. However, because
% of issues like this, it may be the safest practice to put all your
% \label just after \caption rather than within \caption{}.
%
% Reminder: the "draftcls" or "draftclsnofoot", not "draft", class
% option should be used if it is desired that the figures are to be
% displayed while in draft mode.
%
%\begin{figure}[!t]
%\centering
%\includegraphics[width=2.5in]{myfigure}
% where an .eps filename suffix will be assumed under latex,
% and a .pdf suffix will be assumed for pdflatex; or what has been declared
% via \DeclareGraphicsExtensions.
%\caption{Simulation Results}
%\label{fig_sim}
%\end{figure}

% Note that IEEE typically puts floats only at the top, even when this
% results in a large percentage of a column being occupied by floats.


% An example of a double column floating figure using two subfigures.
% (The subfig.sty package must be loaded for this to work.)
% The subfigure \label commands are set within each subfloat command, the
% \label for the overall figure must come after \caption.
% \hfil must be used as a separator to get equal spacing.
% The subfigure.sty package works much the same way, except \subfigure is
% used instead of \subfloat.
%
%\begin{figure*}[!t]
%\centerline{\subfloat[Case I]\includegraphics[width=2.5in]{subfigcase1}%
%\label{fig_first_case}}
%\hfil
%\subfloat[Case II]{\includegraphics[width=2.5in]{subfigcase2}%
%\label{fig_second_case}}}
%\caption{Simulation results}
%\label{fig_sim}
%\end{figure*}
%
% Note that often IEEE papers with subfigures do not employ subfigure
% captions (using the optional argument to \subfloat), but instead will
% reference/describe all of them (a), (b), etc., within the main caption.


% An example of a floating table. Note that, for IEEE style tables, the
% \caption command should come BEFORE the table. Table text will default to
% \footnotesize as IEEE normally uses this smaller font for tables.
% The \label must come after \caption as always.
%
%\begin{table}[!t]
%% increase table row spacing, adjust to taste
%\renewcommand{\arraystretch}{1.3}
% if using array.sty, it might be a good idea to tweak the value of
% \extrarowheight as needed to properly center the text within the cells
%\caption{An Example of a Table}
%\label{table_example}
%\centering
%% Some packages, such as MDW tools, offer better commands for making tables
%% than the plain LaTeX2e tabular which is used here.
%\begin{tabular}{|c||c|}
%\hline
%One & Two\\
%\hline
%Three & Four\\
%\hline
%\end{tabular}
%\end{table}


% Note that IEEE does not put floats in the very first column - or typically
% anywhere on the first page for that matter. Also, in-text middle ("here")
% positioning is not used. Most IEEE journals use top floats exclusively.
% Note that, LaTeX2e, unlike IEEE journals, places footnotes above bottom
% floats. This can be corrected via the \fnbelowfloat command of the
% stfloats package.



%\section{Conclusion}
%The conclusion goes here.





% if have a single appendix:
%\appendix[Proof of the Zonklar Equations]
% or
%\appendix  % for no appendix heading
% do not use \section anymore after \appendix, only \section*
% is possibly needed

% use appendices with more than one appendix
% then use \section to start each appendix
% you must declare a \section before using any
% \subsection or using \label (\appendices by itself
% starts a section numbered zero.)
%


%\appendices
%\section{Proof of the First Zonklar Equation}
%Appendix one text goes here.

% you can choose not to have a title for an appendix
% if you want by leaving the argument blank
%\section{}
%Appendix two text goes here.


% use section* for acknowledgement
%\section*{Acknowledgment}


%The authors would like to thank...


% Can use something like this to put references on a page
% by themselves when using endfloat and the captionsoff option.
%\ifCLASSOPTIONcaptionsoff
%  \newpage
%\fi



% trigger a \newpage just before the given reference
% number - used to balance the columns on the last page
% adjust value as needed - may need to be readjusted if
% the document is modified later
%\IEEEtriggeratref{8}
% The "triggered" command can be changed if desired:
%\IEEEtriggercmd{\enlargethispage{-5in}}

% references section

% can use a bibliography generated by BibTeX as a .bbl file
% BibTeX documentation can be easily obtained at:
% http://www.ctan.org/tex-archive/biblio/bibtex/contrib/doc/
% The IEEEtran BibTeX style support page is at:
% http://www.michaelshell.org/tex/ieeetran/bibtex/
\bibliographystyle{IEEEtran}
% argument is your BibTeX string definitions and bibliography database(s)
\bibliography{IEEEabrv,my_bib}
%
% <OR> manually copy in the resultant .bbl file
% set second argument of \begin to the number of references
% (used to reserve space for the reference number labels box)
%\begin{thebibliography}{1}

%\bibitem{IEEEhowto:kopka}
%H.~Kopka and P.~W. Daly, \emph{A Guide to \LaTeX}, 3rd~ed.\hskip 1em plus
%  0.5em minus 0.4em\relax Harlow, England: Addison-Wesley, 1999.

%\end{thebibliography}

% biography section
%
% If you have an EPS/PDF photo (graphicx package needed) extra braces are
% needed around the contents of the optional argument to biography to prevent
% the LaTeX parser from getting confused when it sees the complicated
% \includegraphics command within an optional argument. (You could create
% your own custom macro containing the \includegraphics command to make things
% simpler here.)
%\begin{biography}[{\includegraphics[width=1in,height=1.25in,clip,keepaspectratio]{mshell}}]{Michael Shell}
% or if you just want to reserve a space for a photo:

%\begin{IEEEbiography}{Michael Shell}
%Biography text here.
%\end{IEEEbiography}

% if you will not have a photo at all:
%\begin{IEEEbiographynophoto}{John Doe}
%Biography text here.
%\end{IEEEbiographynophoto}

% insert where needed to balance the two columns on the last page with
% biographies
%\newpage

%\begin{IEEEbiographynophoto}{Jane Doe}
%Biography text here.
%\end{IEEEbiographynophoto}

% You can push biographies down or up by placing
% a \vfill before or after them. The appropriate
% use of \vfill depends on what kind of text is
% on the last page and whether or not the columns
% are being equalized.

%\vfill

% Can be used to pull up biographies so that the bottom of the last one
% is flush with the other column.
%\enlargethispage{-5in}



% that's all folks
\end{document}
