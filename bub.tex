
%% bare_jrnl.tex
%% V1.3
%% 2007/01/11
%% by Michael Shell
%% see http://www.michaelshell.org/
%% for current contact information.
%%
%% This is a skeleton file demonstrating the use of IEEEtran.cls
%% (requires IEEEtran.cls version 1.7 or later) with an IEEE journal paper.
%%
%% Support sites:
%% http://www.michaelshell.org/tex/ieeetran/
%% http://www.ctan.org/tex-archive/macros/latex/contrib/IEEEtran/
%% and
%% http://www.ieee.org/



% *** Authors should verify (and, if needed, correct) their LaTeX system  ***
% *** with the testflow diagnostic prior to trusting their LaTeX platform ***
% *** with production work. IEEE's font choices can trigger bugs that do  ***
% *** not appear when using other class files.                            ***
% The testflow support page is at:
% http://www.michaelshell.org/tex/testflow/


%%*************************************************************************
%% Legal Notice:
%% This code is offered as-is without any warranty either expressed or
%% implied; without even the implied warranty of MERCHANTABILITY or
%% FITNESS FOR A PARTICULAR PURPOSE! 
%% User assumes all risk.
%% In no event shall IEEE or any contributor to this code be liable for
%% any damages or losses, including, but not limited to, incidental,
%% consequential, or any other damages, resulting from the use or misuse
%% of any information contained here.
%%
%% All comments are the opinions of their respective authors and are not
%% necessarily endorsed by the IEEE.
%%
%% This work is distributed under the LaTeX Project Public License (LPPL)
%% ( http://www.latex-project.org/ ) version 1.3, and may be freely used,
%% distributed and modified. A copy of the LPPL, version 1.3, is included
%% in the base LaTeX documentation of all distributions of LaTeX released
%% 2003/12/01 or later.
%% Retain all contribution notices and credits.
%% ** Modified files should be clearly indicated as such, including  **
%% ** renaming them and changing author support contact information. **
%%
%% File list of work: IEEEtran.cls, IEEEtran_HOWTO.pdf, bare_adv.tex,
%%                    bare_conf.tex, bare_jrnl.tex, bare_jrnl_compsoc.tex
%%*************************************************************************

% Note that the a4paper option is mainly intended so that authors in
% countries using A4 can easily print to A4 and see how their papers will
% look in print - the typesetting of the document will not typically be
% affected with changes in paper size (but the bottom and side margins will).
% Use the testflow package mentioned above to verify correct handling of
% both paper sizes by the user's LaTeX system.
%
% Also note that the "draftcls" or "draftclsnofoot", not "draft", option
% should be used if it is desired that the figures are to be displayed in
% draft mode.
%
\documentclass[journal]{IEEEtran}
%
% If IEEEtran.cls has not been installed into the LaTeX system files,
% manually specify the path to it like:
% \documentclass[journal]{../sty/IEEEtran}





% Some very useful LaTeX packages include:
% (uncomment the ones you want to load)


% *** MISC UTILITY PACKAGES ***
%
%\usepackage{ifpdf}
% Heiko Oberdiek's ifpdf.sty is very useful if you need conditional
% compilation based on whether the output is pdf or dvi.
% usage:
% \ifpdf
%   % pdf code
% \else
%   % dvi code
% \fi
% The latest version of ifpdf.sty can be obtained from:
% http://www.ctan.org/tex-archive/macros/latex/contrib/oberdiek/
% Also, note that IEEEtran.cls V1.7 and later provides a builtin
% \ifCLASSINFOpdf conditional that works the same way.
% When switching from latex to pdflatex and vice-versa, the compiler may
% have to be run twice to clear warning/error messages.






% *** CITATION PACKAGES ***
%
\usepackage{cite}
% cite.sty was written by Donald Arseneau
% V1.6 and later of IEEEtran pre-defines the format of the cite.sty package
% \cite{} output to follow that of IEEE. Loading the cite package will
% result in citation numbers being automatically sorted and properly
% "compressed/ranged". e.g., [1], [9], [2], [7], [5], [6] without using
% cite.sty will become [1], [2], [5]--[7], [9] using cite.sty. cite.sty's
% \cite will automatically add leading space, if needed. Use cite.sty's
% noadjust option (cite.sty V3.8 and later) if you want to turn this off.
% cite.sty is already installed on most LaTeX systems. Be sure and use
% version 4.0 (2003-05-27) and later if using hyperref.sty. cite.sty does
% not currently provide for hyperlinked citations.
% The latest version can be obtained at:
% http://www.ctan.org/tex-archive/macros/latex/contrib/cite/
% The documentation is contained in the cite.sty file itself.






% *** GRAPHICS RELATED PACKAGES ***
%
\ifCLASSINFOpdf
  % \usepackage[pdftex]{graphicx}
  % declare the path(s) where your graphic files are
  % \graphicspath{{../pdf/}{../jpeg/}}
  % and their extensions so you won't have to specify these with
  % every instance of \includegraphics
  % \DeclareGraphicsExtensions{.pdf,.jpeg,.png}
\else
  % or other class option (dvipsone, dvipdf, if not using dvips). graphicx
  % will default to the driver specified in the system graphics.cfg if no
  % driver is specified.
   \usepackage[dvips]{graphicx}
  % declare the path(s) where your graphic files are
  % \graphicspath{{../eps/}}
  % and their extensions so you won't have to specify these with
  % every instance of \includegraphics
  % \DeclareGraphicsExtensions{.eps}
\fi
% graphicx was written by David Carlisle and Sebastian Rahtz. It is
% required if you want graphics, photos, etc. graphicx.sty is already
% installed on most LaTeX systems. The latest version and documentation can
% be obtained at: 
% http://www.ctan.org/tex-archive/macros/latex/required/graphics/
% Another good source of documentation is "Using Imported Graphics in
% LaTeX2e" by Keith Reckdahl which can be found as epslatex.ps or
% epslatex.pdf at: http://www.ctan.org/tex-archive/info/
%
% latex, and pdflatex in dvi mode, support graphics in encapsulated
% postscript (.eps) format. pdflatex in pdf mode supports graphics
% in .pdf, .jpeg, .png and .mps (metapost) formats. Users should ensure
% that all non-photo figures use a vector format (.eps, .pdf, .mps) and
% not a bitmapped formats (.jpeg, .png). IEEE frowns on bitmapped formats
% which can result in "jaggedy"/blurry rendering of lines and letters as
% well as large increases in file sizes.
%
% You can find documentation about the pdfTeX application at:
% http://www.tug.org/applications/pdftex





% *** MATH PACKAGES ***
%
\usepackage[cmex10]{amsmath}
\usepackage{amsfonts}
% A popular package from the American Mathematical Society that provides
% many useful and powerful commands for dealing with mathematics. If using
% it, be sure to load this package with the cmex10 option to ensure that
% only type 1 fonts will utilized at all point sizes. Without this option,
% it is possible that some math symbols, particularly those within
% footnotes, will be rendered in bitmap form which will result in a
% document that can not be IEEE Xplore compliant!
%
% Also, note that the amsmath package sets \interdisplaylinepenalty to 10000
% thus preventing page breaks from occurring within multiline equations. Use:
%\interdisplaylinepenalty=2500
% after loading amsmath to restore such page breaks as IEEEtran.cls normally
% does. amsmath.sty is already installed on most LaTeX systems. The latest
% version and documentation can be obtained at:
% http://www.ctan.org/tex-archive/macros/latex/required/amslatex/math/





% *** SPECIALIZED LIST PACKAGES ***
%
%\usepackage{algorithmic}
% algorithmic.sty was written by Peter Williams and Rogerio Brito.
% This package provides an algorithmic environment fo describing algorithms.
% You can use the algorithmic environment in-text or within a figure
% environment to provide for a floating algorithm. Do NOT use the algorithm
% floating environment provided by algorithm.sty (by the same authors) or
% algorithm2e.sty (by Christophe Fiorio) as IEEE does not use dedicated
% algorithm float types and packages that provide these will not provide
% correct IEEE style captions. The latest version and documentation of
% algorithmic.sty can be obtained at:
% http://www.ctan.org/tex-archive/macros/latex/contrib/algorithms/
% There is also a support site at:
% http://algorithms.berlios.de/index.html
% Also of interest may be the (relatively newer and more customizable)
% algorithmicx.sty package by Szasz Janos:
% http://www.ctan.org/tex-archive/macros/latex/contrib/algorithmicx/




% *** ALIGNMENT PACKAGES ***
%
%\usepackage{array}
% Frank Mittelbach's and David Carlisle's array.sty patches and improves
% the standard LaTeX2e array and tabular environments to provide better
% appearance and additional user controls. As the default LaTeX2e table
% generation code is lacking to the point of almost being broken with
% respect to the quality of the end results, all users are strongly
% advised to use an enhanced (at the very least that provided by array.sty)
% set of table tools. array.sty is already installed on most systems. The
% latest version and documentation can be obtained at:
% http://www.ctan.org/tex-archive/macros/latex/required/tools/


%\usepackage{mdwmath}
%\usepackage{mdwtab}
% Also highly recommended is Mark Wooding's extremely powerful MDW tools,
% especially mdwmath.sty and mdwtab.sty which are used to format equations
% and tables, respectively. The MDWtools set is already installed on most
% LaTeX systems. The lastest version and documentation is available at:
% http://www.ctan.org/tex-archive/macros/latex/contrib/mdwtools/


% IEEEtran contains the IEEEeqnarray family of commands that can be used to
% generate multiline equations as well as matrices, tables, etc., of high
% quality.


%\usepackage{eqparbox}
% Also of notable interest is Scott Pakin's eqparbox package for creating
% (automatically sized) equal width boxes - aka "natural width parboxes".
% Available at:
% http://www.ctan.org/tex-archive/macros/latex/contrib/eqparbox/





% *** SUBFIGURE PACKAGES ***
\usepackage[tight,footnotesize]{subfigure}
% subfigure.sty was written by Steven Douglas Cochran. This package makes it
% easy to put subfigures in your figures. e.g., "Figure 1a and 1b". For IEEE
% work, it is a good idea to load it with the tight package option to reduce
% the amount of white space around the subfigures. subfigure.sty is already
% installed on most LaTeX systems. The latest version and documentation can
% be obtained at:
% http://www.ctan.org/tex-archive/obsolete/macros/latex/contrib/subfigure/
% subfigure.sty has been superceeded by subfig.sty.



%\usepackage[caption=false]{caption}
%\usepackage[font=footnotesize,caption=false]{subfig}
% subfig.sty, also written by Steven Douglas Cochran, is the modern
% replacement for subfigure.sty. However, subfig.sty requires and
% automatically loads Axel Sommerfeldt's caption.sty which will override
% IEEEtran.cls handling of captions and this will result in nonIEEE style
% figure/table captions. To prevent this problem, be sure and preload
% caption.sty with its "caption=false" package option. This is will preserve
% IEEEtran.cls handing of captions. Version 1.3 (2005/06/28) and later 
% (recommended due to many improvements over 1.2) of subfig.sty supports
% the caption=false option directly:
%\usepackage[caption=false,font=footnotesize]{subfig}
%
% The latest version and documentation can be obtained at:
% http://www.ctan.org/tex-archive/macros/latex/contrib/subfig/
% The latest version and documentation of caption.sty can be obtained at:
% http://www.ctan.org/tex-archive/macros/latex/contrib/caption/




% *** FLOAT PACKAGES ***
%
%\usepackage{fixltx2e}
% fixltx2e, the successor to the earlier fix2col.sty, was written by
% Frank Mittelbach and David Carlisle. This package corrects a few problems
% in the LaTeX2e kernel, the most notable of which is that in current
% LaTeX2e releases, the ordering of single and double column floats is not
% guaranteed to be preserved. Thus, an unpatched LaTeX2e can allow a
% single column figure to be placed prior to an earlier double column
% figure. The latest version and documentation can be found at:
% http://www.ctan.org/tex-archive/macros/latex/base/



%\usepackage{stfloats}
% stfloats.sty was written by Sigitas Tolusis. This package gives LaTeX2e
% the ability to do double column floats at the bottom of the page as well
% as the top. (e.g., "\begin{figure*}[!b]" is not normally possible in
% LaTeX2e). It also provides a command:
%\fnbelowfloat
% to enable the placement of footnotes below bottom floats (the standard
% LaTeX2e kernel puts them above bottom floats). This is an invasive package
% which rewrites many portions of the LaTeX2e float routines. It may not work
% with other packages that modify the LaTeX2e float routines. The latest
% version and documentation can be obtained at:
% http://www.ctan.org/tex-archive/macros/latex/contrib/sttools/
% Documentation is contained in the stfloats.sty comments as well as in the
% presfull.pdf file. Do not use the stfloats baselinefloat ability as IEEE
% does not allow \baselineskip to stretch. Authors submitting work to the
% IEEE should note that IEEE rarely uses double column equations and
% that authors should try to avoid such use. Do not be tempted to use the
% cuted.sty or midfloat.sty packages (also by Sigitas Tolusis) as IEEE does
% not format its papers in such ways.


%\ifCLASSOPTIONcaptionsoff
%  \usepackage[nomarkers]{endfloat}
% \let\MYoriglatexcaption\caption
% \renewcommand{\caption}[2][\relax]{\MYoriglatexcaption[#2]{#2}}
%\fi
% endfloat.sty was written by James Darrell McCauley and Jeff Goldberg.
% This package may be useful when used in conjunction with IEEEtran.cls'
% captionsoff option. Some IEEE journals/societies require that submissions
% have lists of figures/tables at the end of the paper and that
% figures/tables without any captions are placed on a page by themselves at
% the end of the document. If needed, the draftcls IEEEtran class option or
% \CLASSINPUTbaselinestretch interface can be used to increase the line
% spacing as well. Be sure and use the nomarkers option of endfloat to
% prevent endfloat from "marking" where the figures would have been placed
% in the text. The two hack lines of code above are a slight modification of
% that suggested by in the endfloat docs (section 8.3.1) to ensure that
% the full captions always appear in the list of figures/tables - even if
% the user used the short optional argument of \caption[]{}.
% IEEE papers do not typically make use of \caption[]'s optional argument,
% so this should not be an issue. A similar trick can be used to disable
% captions of packages such as subfig.sty that lack options to turn off
% the subcaptions:
% For subfig.sty:
% \let\MYorigsubfloat\subfloat
% \renewcommand{\subfloat}[2][\relax]{\MYorigsubfloat[]{#2}}
% For subfigure.sty:
% \let\MYorigsubfigure\subfigure
% \renewcommand{\subfigure}[2][\relax]{\MYorigsubfigure[]{#2}}
% However, the above trick will not work if both optional arguments of
% the \subfloat/subfig command are used. Furthermore, there needs to be a
% description of each subfigure *somewhere* and endfloat does not add
% subfigure captions to its list of figures. Thus, the best approach is to
% avoid the use of subfigure captions (many IEEE journals avoid them anyway)
% and instead reference/explain all the subfigures within the main caption.
% The latest version of endfloat.sty and its documentation can obtained at:
% http://www.ctan.org/tex-archive/macros/latex/contrib/endfloat/
%
% The IEEEtran \ifCLASSOPTIONcaptionsoff conditional can also be used
% later in the document, say, to conditionally put the References on a 
% page by themselves.





% *** PDF, URL AND HYPERLINK PACKAGES ***
%
%\usepackage{url}
% url.sty was written by Donald Arseneau. It provides better support for
% handling and breaking URLs. url.sty is already installed on most LaTeX
% systems. The latest version can be obtained at:
% http://www.ctan.org/tex-archive/macros/latex/contrib/misc/
% Read the url.sty source comments for usage information. Basically,
% \url{my_url_here}.





% *** Do not adjust lengths that control margins, column widths, etc. ***
% *** Do not use packages that alter fonts (such as pslatex).         ***
% There should be no need to do such things with IEEEtran.cls V1.6 and later.
% (Unless specifically asked to do so by the journal or conference you plan
% to submit to, of course. )


% correct bad hyphenation here
\hyphenation{op-tical net-works semi-conduc-tor}


\begin{document}
%
% paper title
% can use linebreaks \\ within to get better formatting as desired
\title{Bottom-up Broadband: Bringing the Open Source Spirit to Networking Initiatives}
%
%
% author names and IEEE memberships
% note positions of commas and nonbreaking spaces ( ~ ) LaTeX will not break
% a structure at a ~ so this keeps an author's name from being broken across
% two lines.
% use \thanks{} to gain access to the first footnote area
% a separate \thanks must be used for each paragraph as LaTeX2e's \thanks
% was not built to handle multiple paragraphs
%

\author{
	Name~Surname,~%\IEEEmembership{Member,~IEEE,}
    Name~Surname,~%\IEEEmembership{Member,~IEEE,}
    Name~Surname,~%\IEEEmembership{Member,~IEEE,}
    Name~Surname,~%\IEEEmembership{Member,~IEEE,}
    and~Name~Malone%~\IEEEmembership{Life~Fellow,~IEEE}% <-this % stops a space
\thanks{The author are with bub univeristy}
}


% note the % following the last \IEEEmembership and also \thanks - 
% these prevent an unwanted space from occurring between the last author name
% and the end of the author line. i.e., if you had this:
% 
% \author{....lastname \thanks{...} \thanks{...} }
%                     ^------------^------------^----Do not want these spaces!
%
% a space would be appended to the last name and could cause every name on that
N% line to be shifted left slightly. This is one of those "LaTeX things". For
% instance, "\textbf{A} \textbf{B}" will typeset as "A B" not "AB". To get
% "AB" then you have to do: "\textbf{A}\textbf{B}"
% \thanks is no different in this regard, so shield the last } of each \thanks
% that ends a line with a % and do not let a space in before the next \thanks.
% Spaces after \IEEEmembership other than the last one are OK (and needed) as
% you are supposed to have spaces between the names. For what it is worth,
% this is a minor point as most people would not even notice if the said evil
% space somehow managed to creep in.



% The paper headers
%\markboth{Journal of \LaTeX\ Class Files,~Vol.~6, No.~1, January~2007}%
%{Shell \MakeLowercase{\textit{et al.}}: Bare Demo of IEEEtran.cls for Journals}
% The only time the second header will appear is for the odd numbered pages
% after the title page when using the twoside option.
% 
% *** Note that you probably will NOT want to include the author's ***
% *** name in the headers of peer review papers.                   ***
% You can use \ifCLASSOPTIONpeerreview for conditional compilation here if
% you desire.




% If you want to put a publisher's ID mark on the page you can do it like
% this:
%\IEEEpubid{0000--0000/00\$00.00~\copyright~2007 IEEE}
% Remember, if you use this you must call \IEEEpubidadjcol in the second
% column for its text to clear the IEEEpubid mark.



% use for special paper notices
%\IEEEspecialpapernotice{(Invited Paper)}




% make the title area
\maketitle


\begin{abstract}
%\boldmath
Open Source software development exemplifies a collaborative production model that comes with several advantages.
This advantages include efficiency, quality, affordability and educational value.
It is not surprising that this model is being applied in other contexts to obtain similar advantages.
As an example, Creative Commons content and the collaborative enciclopedia Wikipedia reproduce with great success the production model of Open Source software.

In this article we focus on networks and networking initiatives that draw on similar openness principles.
We differentiate between Open Access Network which encourage a layered network deployment infrastructure that lowers entry barriers and encourages specialization, and Free (Libre) Networks in which the network is under control of the community.
\end{abstract}
% IEEEtran.cls defaults to using nonbold math in the Abstract.
% This preserves the distinction between vectors and scalars. However,
% if the journal you are submitting to favors bold math in the abstract,
% then you can use LaTeX's standard command \boldmath at the very start
% of the abstract to achieve this. Many IEEE journals frown on math
% in the abstract anyway.

% Note that keywords are not normally used for peerreview papers.
%\begin{IEEEkeywords}
% media access control, WLAN, collision-free schedule.
%Slotted Aloha, game theory, contention control, media access control.
%\end{IEEEkeywords}






% For peer review papers, you can put extra information on the cover
% page as needed:
% \ifCLASSOPTIONpeerreview
% \begin{center} \bfseries EDICS Category: 3-BBND \end{center}
% \fi
%
% For peerreview papers, this IEEEtran command inserts a page break and
% creates the second title. It will be ignored for other modes.
\IEEEpeerreviewmaketitle



\section{Introduction}
% The very first letter is a 2 line initial drop letter followed
% by the rest of the first word in caps.
% 
% form to use if the first word consists of a single letter:
% \IEEEPARstart{A}{demo} file is ....
% 
% form to use if you need the single drop letter followed by
% normal text (unknown if ever used by IEEE):
% \IEEEPARstart{A}{}demo file is ....
% 
% Some journals put the first two words in caps:
% \IEEEPARstart{T}{his demo} file is ....
% 
% Here we have the typical use of a "T" for an initial drop letter
% and "HIS" in caps to complete the first word.
\IEEEPARstart{T}{his} is the first sentence of the article, which I hope is long enough \cite{abramson2009asw}.

\section{Peer-to-Peer Production Model}
The peer-to-peer production model relies on horizontal symmetric relations.
It is in contrast to the hierarchical boss-to-subordinate model and the consumer-producer model.
This model is well understood by the networking community as there is a large experience wit peer-to-peer distributed applications in which the participating nodes take turns in assuming the roles of client and servers.
In the networked applications realm, peer-to-peer offers advantages such as scalability and resilience, and also some challenges such as bootstraping or providing quality-of-service guarantees.

When applied to the organization of people

\section{Open Access Networks}
\section{Free (Libre) Networks}
\section{The Bottom-up Broadband Project}
\section{Collision avoidance (CA) and enhanced collision avoidance (ECA) in CSMA networks}
\label{sec:eca}

\begin{figure*}[!t]
\centering
\subfigure[CSMA/CA]{\includegraphics[width=5.5in]{figures/csma_ca}%
\label{fig:csma_ca}}
\subfigure[CSMA/ECA]{\includegraphics[width=5.5in]{figures/csma_eca}%
\label{fig:csma_eca}}
\caption{Examples of contention in which two wireless stations compete for channel access. The rounded boxes represent transmissions and the numbers are the backoff counters. It can be observed that CSMA/ECA attains cyclic collision-free operation after the construction of the schedule (transient convergence).}
\label{fig:ca_vs_eca}
\end{figure*}

Most of the currently deployed WLANs are compliant with the IEEE 802.11 standard and rely on CSMA/CA to share the channel time.
Thanks to the carrier sense capabilities of the CSMA stations, channel time can be divided in variable length slots.
We classify the slots as either empty, if no station transmits, or busy, if one or more stations transmit.
Among busy slots, we differentiate between successful slots, when there is a single transmission, and collision slots, when multiple stations simultaneously transmit.
Empty slots are relatively short and of constant duration, which is specified by the standard, and  busy slots are of variable length. 
As an example, the empty slot duration for IEEE 802.11b is 20 $\mu$s and a busy slot can be 1200 $\mu$s long.

Since the stations can use carrier sensing to detect the end of a transmission, it is possible to synchronize the nodes to the end of variable length transmissions.
The fact that the empty slots can be orders of magnitude shorter than the busy slots represents a performance gain over those approaches in which the slot size is fixed and constant.

In wired networks, it is possible for the nodes involved in a collision to detect the collision while it is taking place and immediately stop transmitting.
This technique is called CSMA with collision detection (CSMA/CD) and keeps the duration of collision slots very short.

In contrast, wireless devices do not have the possibility to detect a collision while they are transmitting.
In fact, wireless stations can only learn about the success (or failure) of a transmission by means of feedback (or lack thereof) from the receiver.
For this reason, the length of a collision slot is approximately equal to the length of the longest transmission involved in the collision.

To reduce the likelihood of collisions, in CSMA/CA, the channel is divided into slots and transmissions are synchronized to slot borders and preceded by a random backoff.
In particular, stations performing backoff set a backoff counter to a randomly chosen value and decrement it by one at every slot.
The transmission occurs when the backoff counter reaches zero.

\subsection{The construction of a collision-free schedule for two contending stations}
CSMA/ECA is simply a subtle variant of the protocol described above.
The only difference between CSMA/CA and CSMA/ECA is that the latter uses a deterministic backoff after successful transmissions.
This deterministic backoff is constant and is the same for all the stations.
As a result, two stations that successfully transmit in two different slots will not collide with each other in their next transmission attempt.

Imagine that two stations STA 1 and STA 2 successfully transmit in two different slots ($X$ and $Y$),  and then they both backoff for the same number of slots $V$.
Their next transmission attempt occurs at slot $X+V$ and $Y+V$, which are different since $X$ and $Y$ are different.

The behavior of CSMA/CA and CSMA/ECA for a network of two nodes is depicted in Fig.~\ref{fig:ca_vs_eca}.
Fig.~\ref{fig:csma_ca} represents two stations competing for the channel using CSMA/CA.
The channel time is slotted and some slots are empty while others are busy with successes or collisions.
If realistic channels are considered, it is also possible that a busy slot contains a transmission that cannot be decoded due to unfavorable channel conditions.
Nevertheless, in this tutorial introduction, we will consider only an ideal channel that does not introduce errors.

The figure is not to scale for the ease of representation.
In reality, the busy slots are much longer than the empty ones. 
The figure also shows the backoff value of each of the two competing stations in each slot, and the tiny arrows indicate whether the backoff is randomly or deterministically selected.

It can be observed that the backoff value is decremented by one in every slot and that a station transmits when its backoff counter reaches zero.
After a transmission, each CSMA/CA station randomly chooses a new backoff value.

In the present example we assume that, after completing a transmission, each station has another packet to transmit.
In the literature, this particular assumption is often referred to as saturation condition (e.g., \cite{he2009srb,barcelo2010fcc,fang2011dlm,barcelo2011tcf}).
We will keep the saturation assumption in the remainder of the paper, although in the next section we will mention references that address the non-saturation scenario.

The CSMA/CA stations in Fig.~\ref{fig:csma_ca} always use a random backoff, which means that they are always exposed to a collision probability greater than zero.
It is useful to compare the behavior of CSMA/CA in Fig.~\ref{fig:csma_ca} to the behavior of CSMA/ECA in Fig.~\ref{fig:csma_eca}.
The initial behavior is exactly the same for the two protocols: a collision occurs and a random backoff is selected.
However, after the first successful transmission of STA 1 we can observe that the CSMA/ECA station deterministically chooses its backoff value.
The same occurs after the first successful transmission of STA 2.
The fact that the stations have successfully transmitted in different slots and use the same deterministic backoff value guarantees that these two stations will not collide with each other in their next transmission attempt.
From this point on, the behavior of the system is collision-free, deterministic, cyclic and fair.
The cycle length is indicated in the figure, and it is easy to observe that the behavior of the system in the second cycle is exactly the same as in the first cycle.
There is no need for a global agreement about which is the first slot of a cycle.
For example, each station can consider its own transmitting slot as the first slot of the cycle.

\subsection{Generalization to a larger number of contenders}
The general rule is that collision-free operation is reached after all the contending stations successfully transmit within a single cycle duration.
To better understand the construction of the collision-free schedule, it is useful to look at an example with more than two contending stations.
In order to depict the contention for the channel when the number of contenders is high, we will need a more compact representation such as the one used in Fig.~\ref{fig:ca_vs_eca_compact}.
For convenience, we draw all the slots with equal length.
Each slot is numbered and the transmissions of the stations are represented as disks in the slots.
There are six different stations competing for the channel and the hatching pattern of each disk identifies the transmitting station.

\begin{figure*}[!t]
\centering
\subfigure[CSMA/CA]{\includegraphics[width=2.5in]{figures/csma_ca_compact}%
\label{fig:csma_ca_compact}}
\hspace{25mm}
\subfigure[CSMA/ECA]{\includegraphics[width=2.5in]{figures/csma_eca_compact}%
\label{fig:csma_eca_compact}}
\caption{A compact representation of contention in which six wireless stations compete for channel access. The disks represent the transmissions of the stations and the  patterns are used to identify the station that transmitted. The construction of the collision-free schedule in CSMA/ECA finishes when all the stations successfully transmit in the same cycle.}
\label{fig:ca_vs_eca_compact}
\end{figure*}

As in the previous example in Fig.~\ref{fig:csma_eca}, in the CSMA/ECA example in Fig.~\ref{fig:csma_eca_compact} the stations use a deterministic backoff after successful transmissions.
For convenience, the slots have been arranged in such a way that a deterministic backoff is represented by a new transmission in the same column of the following row.
As an example, the CSMA/ECA station that successfully transmits in slot 1 transmits again in slot 17, in the same column.
If we focus on the two CSMA/ECA stations that collide in slot 7, we realize that they use a random backoff which means that the new transmissions will probably end up in a different column.
In this particular example, the colliding stations in slot 7 retransmit in slot 17 and 27.
In CSMA/ECA, when all the stations successfully transmit in the same cycle, they all stick to the same column.
At this point, the collision-free schedule has already been constructed as we can observe in the last two rows of Fig.~\ref{fig:csma_eca_compact}.

The construction of the collision-free schedule results in significant performance gains in terms of throughput, as we will see in the next section.
Because the deterministic stations may only collide with random stations and not with one another, CSMA/ECA delivers a performance advantage even before the collision-free schedule is completely constructed.
This means that CSMA/ECA also outperforms CSMA/CA in highly dynamic scenarios in which the stations join and leave the contention.
In the extreme case in which the stations join the contention to transmit a single packet and then they leave, the performance of CSMA/ECA falls back to that of CSMA/CA.

A key aspect of the proposed protocol is that of the schedule length, which is equivalent to the deterministic backoff used after successful transmissions.
If the schedule length is excessively large compared to the number of contenders, the large number of empty slots will slightly penalize the performance.
On the other hand, if the schedule is too short, it will not be possible to accommodate the collision-free operation of all the participants.
As pointed out in \cite{fang2011dlm}, having a schedule that is larger than the number of contenders is better than having one that is shorter.
The reason is that empty slots are much shorter than collision slots and therefore, idle waiting is far less costly than collisions.
We discuss in the next section the possibility to adapt the schedule length in a distributed way.

Even though there are clear similarities between CSMA/ECA and Reservation Aloha, there are also two remarkable differences.
The first one is that in Reservation Aloha the slot size is fixed, while in CSMA/ECA the slot size is variable.
The second difference is that in Reservation Aloha there is slot reservation while in CSMA/ECA there is not.
A station that successfully transmits in CSMA/ECA can suffer a collision in its next transmission attempt, because there is no reservation in place.
Since there is no reservation, a station behaving randomly may choose the same slot as a station that is behaving deterministically.

The lack of reservations in CSMA/ECA makes the protocol very similar to CSMA/CA and allows for the peaceful coexistence of both protocols in the same network.
The similarity of CSMA/CA and CSMA/ECA is also an advantage as it eases the adaptation of current designs to the new protocol.
It is remarkable that the performance advantage of CSMA/ECA does not come at the price of additional signaling or extra overheads.


This section has covered the basic idea that enables the construction of a collision-free schedule in a highly idealized and simplified scenario. If CSMA/ECA is to be considered as a replacement of CSMA/CA, wider and deeper analysis is needed. The following section offers an overview of some contributions in this particular research area.

\section{Mathematical framework, performance evaluation, and refinements}
\label{sec:survey}
In this section we will summarize a small subset of representative contributions to offer an overview of some of the problems and possible enhancements of the basic idea described in the previous section. 

\subsection{Underlying mathematical framework}
Even though CSMA/ECA was initially suggested to prevent collisions in WLANs, the underlying mathematical framework is applicable to various resource allocation problems in the field of wireless networking, such as cognitive radio \cite{khan2013aso}, channel selection and network coding \cite{duffy2011dcs}.
The construction of a collision-free schedule in CSMA/ECA is in fact just an instance of a Constraint Satisfaction Problem (CSP) that the participating entities need to solve without explicit communication.
It is proven in \cite{duffy2011dcs} that the stochastic decentralized CSP solver (which is a generalization of the protocol that we have introduced in the previous section) guarantees that a solution will be found in finite time, if a solution exists.
Furthermore, its performance is competitive with some of the well-known centralized CSP solvers.

\subsection{Distributed adjustment of the cycle length}

Some improvements on the basic idea described in Sec.~\ref{sec:eca} are presented in \cite{fang2011dlm}. 
Namely, it suggests a distributed approach for adjusting the schedule length to accommodate a large number of contenders.
Furthermore, it introduces the concept of stickiness, whereby the stations stick to a deterministic backoff even after a transmission failure, for increased schedule robustness. 

Ideally, the deterministic backoff (which is equivalent to the number of columns in our representation) would be adjusted as a function of the number of contenders.
However, reaching this goal in a distributed fashion without requiring any kind of message exchange and preserving the system's fairness is quite a challenge.
The solution proposed in \cite{fang2011dlm} is elegant and effective:
A station that perceives a high collision probability doubles the deterministic backoff that it uses after successful transmissions.

\begin{figure*}[!t]
\centering
\includegraphics[width=6.0in]{figures/csma_eca_different_backoff}
\caption{Schedule length distributed adaption example. All the stations transmit the same number of packets in each cycle, despite using different schedule lengths.}
\label{fig:csma_eca_different_backoff}
\end{figure*}

The beautiful aspect of this approach is that the station that doubles its deterministic backoff also doubles the number of packets that are transmitted every time that it accesses the medium.
Using this trick, the number of available slots increases without any reduction in throughput.
In the long term, all the stations transmit the same number of packets, independently of their schedule length.
This property makes it possible for different stations to independently adjust their deterministic backoff value while preserving fairness.

As an example, consider the 3-node network shown in Fig.~\ref{fig:csma_eca_different_backoff}, where all three stations have reached a collision-free schedule.
Notice that the schedule length of STA 3 is twice as long as that of the other two stations.
Nevertheless, in terms of fairness, all the stations fairly share the channel.
The stations with short schedules transmit a single packet and the station with the long schedule transmits two packets when it is its turn.
Transmitting more than one packet when accessing the channel is possible and the latest revision of the IEEE 802.11 standard includes the necessary mechanisms for transmitting two or more packets back-to-back (packet aggregation).

\begin{figure}[!t]
\centering
\includegraphics[width=2.5in]{figures/csma_e2ca}
\caption{The use of a deterministic backoff for two consecutive times after each successful transmission (CSMA/E2CA) speeds up the construction of the collision-free schedule.}
\label{fig:csma_e2ca}
\end{figure}


\subsection{Stickiness for faster convergence and increased robustness}

Under ideal conditions, a station using a deterministic backoff may collide only with a station using a random backoff.
If a deterministic station collides with a random station, only one of them needs to choose a random backoff to prevent a new collision of the two stations in their next transmission attempt.
In fact, switching the deterministic station to random behavior only increases the chance of collision with other deterministic stations.
Consequently, the protocol might be improved if deterministic stations kept using a deterministic backoff even after collisions.
The property of using a deterministic backoff after suffering collisions is called ``stickyness'' \cite{fang2011dlm}.

In ideal conditions, this solution has several advantages.
Firstly, it converges faster to collision-free operation, as a station that successfully transmits once never switches back to the random behavior.
Secondly, once a collision-free schedule is built, it is not possible for a channel error to move the system back to the random behavior.
And thirdly, after a collision-free schedule has been built, it is not possible for a new entrant to destroy the schedule.
The new entrant will simply use a random behavior (possibly suffering collisions) until it successfully transmits.

The problem is that real clocks may suffer drifts, that can result in slot misalignment \cite{gong2012asd}.
Two stations using a deterministic backoff may collide if their slot boundaries are not aligned.
This is a very undesirable situation if both stations ``stick'' to the use of deterministic backoff after colliding, as they will likely collide again in the next transmission attempt.

To benefit from the advantages of stickiness while preventing the aforementioned potential pitfall, \cite{fang2011dlm} proposes probabilistic stickiness and \cite{barcelo2011tcf} proposes finite stickiness, in which deterministic stations switch back to random behavior after a given number of consecutive collisions.
CSMA/E2CA in \cite{barcelo2011tcf} moves back to random behavior after two consecutive collisions.

Fig.~\ref{fig:csma_e2ca} illustrates the operation of the protocol when deterministic stations ``stick'' to a deterministic backoff after suffering a collision.
It can be observed that in this example, the convergence to collision-free operation is faster than in Fig.~\ref{fig:csma_eca_compact}.

\subsection{Performance of CSMA/ECA and CSMA/E2CA}

An analytical model of the expected number of slots required to reach collision-free operation is introduced in \cite{he2009srb}.
The paper also presents a comprehensive simulation study which includes realistic ingredients, such as traffic differentiation, carrier-sense errors, and channel errors.
Different performance metrics such as throughput, delay, and collision probability are evaluated, and both saturated and non-saturated traffic is considered.
The authors conclude that a protocol that uses a deterministic backoff after successful transmissions always outperforms the purely random protocol.
Interestingly, the authors report that the implementation of the proposed, protocol in the well-known simulator NS-2 required the change of only three lines of code.
This gives an idea of how similar the proposed protocol is to the legacy one,  and how easy it would be to include the proposed protocol in new devices.

Many performance aspects are covered in \cite{fang2011dlm}, which offers a comparison among different protocols that converge to collision-free operation, and studies the speed of convergence and performance in unsaturated scenarios and in the presence of errors and legacy stations.

Fairness of CSMA/ECA with regard to legacy stations is addressed in \cite{barcelo2010fcc}.
The results show that both protocols are interoperable and can fairly coexist in the same network.
CSMA/ECA stations will experience a slightly better performance than CSMA/CA stations, and the participation of legacy stations prevents the construction of a collision-free schedule.
Nevertheless, it is remarkable that the mix of new and legacy stations attains a better performance than a network in which all the stations follow the legacy protocol.

Backward compatibility is of paramount importance for any improvement to be adopted in WLANs, since there is a large base of deployed hardware that will not be thrown away overnight.
The possibility of CSMA/ECA to peacefully coexist with the previous protocol ensures a smooth transition from one protocol to the other, with a coexistence period in which both protocols will interoperate.

\begin{figure}[!t]
\centering
\includegraphics[width=\linewidth]{figures/performance}
\caption{Performance curves of CSMA/CA and CSMA/E2CA for an increasing number of contenders.}
\label{fig:performance}
\end{figure}

A detailed performance evaluation of CSMA/ECA is offered in \cite{martorell2012pec}.
This paper presents an analytical model and simulations that use realistic channel realizations and Automated Rate Fallback (ARF).
For comparison, results are also presented for CSMA/CA with and without the Request-To-Send/Clear-To-Send (RTS/CTS) four-way handshake.

ARF is a mechanism used to adapt the transmission rate to the channel conditions and simply works by reducing the transmission rate after unsuccessful transmissions.
This approach does not work well with CSMA/CA when collisions occur, as ARF misinterprets all failures as channel errors and reduces the transmission rate, which further worsens the performance when failures are due to collisions.
This problem can be alleviated by using RTS/CTS that can differentiate between collisions and channel errors.
However RTS/CTS penalizes the performance due to the additional overheads.
CSMA/ECA solves the problem by preventing collisions, without adding any additional overhead.
The curves in Fig.~\ref{fig:performance} (reproduced from \cite{martorell2012pec}) show that CSMA/ECA outperforms CSMA/CA and also CSMA/CA with RTS/CTS.

\section{Conclusion} \label{sec:conclusion}
In this paper we have summarized a family of MAC protocols that offer significant performance improvement over CSMA/CA. In its most basic form, CSMA/ECA achieves this performance boost by simply using a deterministic backoff after each successful transmission and reverting back to the CSMA/CA random behavior when a collision is detected. Under certain conditions, this leads to the construction of a collision-free deterministic schedule in a completely distributed fashion. We have then discussed possible variations to CSMA/ECA that can further improve the performance under more realistic conditions. CSMA/ECA and its variants represent a simple evolution of the currently prevalent protocol CSMA/CA, thus offering backward compatibility and fair coexistence with already deployed hardware. 

Similar techniques can be used to address other problems in wireless networking, such as channel assignment, spreading
code assignment, and channel sensing-order assignment in cognitive radio.

% An example of a floating figure using the graphicx package.
% Note that \label must occur AFTER (or within) \caption.
% For figures, \caption should occur after the \includegraphics.
% Note that IEEEtran v1.7 and later has special internal code that
% is designed to preserve the operation of \label within \caption
% even when the captionsoff option is in effect. However, because
% of issues like this, it may be the safest practice to put all your
% \label just after \caption rather than within \caption{}.
%
% Reminder: the "draftcls" or "draftclsnofoot", not "draft", class
% option should be used if it is desired that the figures are to be
% displayed while in draft mode.
%
%\begin{figure}[!t]
%\centering
%\includegraphics[width=2.5in]{myfigure}
% where an .eps filename suffix will be assumed under latex, 
% and a .pdf suffix will be assumed for pdflatex; or what has been declared
% via \DeclareGraphicsExtensions.
%\caption{Simulation Results}
%\label{fig_sim}
%\end{figure}

% Note that IEEE typically puts floats only at the top, even when this
% results in a large percentage of a column being occupied by floats.


% An example of a double column floating figure using two subfigures.
% (The subfig.sty package must be loaded for this to work.)
% The subfigure \label commands are set within each subfloat command, the
% \label for the overall figure must come after \caption.
% \hfil must be used as a separator to get equal spacing.
% The subfigure.sty package works much the same way, except \subfigure is
% used instead of \subfloat.
%
%\begin{figure*}[!t]
%\centerline{\subfloat[Case I]\includegraphics[width=2.5in]{subfigcase1}%
%\label{fig_first_case}}
%\hfil
%\subfloat[Case II]{\includegraphics[width=2.5in]{subfigcase2}%
%\label{fig_second_case}}}
%\caption{Simulation results}
%\label{fig_sim}
%\end{figure*}
%
% Note that often IEEE papers with subfigures do not employ subfigure
% captions (using the optional argument to \subfloat), but instead will
% reference/describe all of them (a), (b), etc., within the main caption.


% An example of a floating table. Note that, for IEEE style tables, the 
% \caption command should come BEFORE the table. Table text will default to
% \footnotesize as IEEE normally uses this smaller font for tables.
% The \label must come after \caption as always.
%
%\begin{table}[!t]
%% increase table row spacing, adjust to taste
%\renewcommand{\arraystretch}{1.3}
% if using array.sty, it might be a good idea to tweak the value of
% \extrarowheight as needed to properly center the text within the cells
%\caption{An Example of a Table}
%\label{table_example}
%\centering
%% Some packages, such as MDW tools, offer better commands for making tables
%% than the plain LaTeX2e tabular which is used here.
%\begin{tabular}{|c||c|}
%\hline
%One & Two\\
%\hline
%Three & Four\\
%\hline
%\end{tabular}
%\end{table}


% Note that IEEE does not put floats in the very first column - or typically
% anywhere on the first page for that matter. Also, in-text middle ("here")
% positioning is not used. Most IEEE journals use top floats exclusively.
% Note that, LaTeX2e, unlike IEEE journals, places footnotes above bottom
% floats. This can be corrected via the \fnbelowfloat command of the
% stfloats package.



%\section{Conclusion}
%The conclusion goes here.





% if have a single appendix:
%\appendix[Proof of the Zonklar Equations]
% or
%\appendix  % for no appendix heading
% do not use \section anymore after \appendix, only \section*
% is possibly needed

% use appendices with more than one appendix
% then use \section to start each appendix
% you must declare a \section before using any
% \subsection or using \label (\appendices by itself
% starts a section numbered zero.)
%


%\appendices
%\section{Proof of the First Zonklar Equation}
%Appendix one text goes here.

% you can choose not to have a title for an appendix
% if you want by leaving the argument blank
%\section{}
%Appendix two text goes here.


% use section* for acknowledgement
%\section*{Acknowledgment}


%The authors would like to thank...


% Can use something like this to put references on a page
% by themselves when using endfloat and the captionsoff option.
%\ifCLASSOPTIONcaptionsoff
%  \newpage
%\fi



% trigger a \newpage just before the given reference
% number - used to balance the columns on the last page
% adjust value as needed - may need to be readjusted if
% the document is modified later
%\IEEEtriggeratref{8}
% The "triggered" command can be changed if desired:
%\IEEEtriggercmd{\enlargethispage{-5in}}

% references section

% can use a bibliography generated by BibTeX as a .bbl file
% BibTeX documentation can be easily obtained at:
% http://www.ctan.org/tex-archive/biblio/bibtex/contrib/doc/
% The IEEEtran BibTeX style support page is at:
% http://www.michaelshell.org/tex/ieeetran/bibtex/
\bibliographystyle{IEEEtran}
% argument is your BibTeX string definitions and bibliography database(s)
\bibliography{IEEEabrv,my_bib}
%
% <OR> manually copy in the resultant .bbl file
% set second argument of \begin to the number of references
% (used to reserve space for the reference number labels box)
%\begin{thebibliography}{1}

%\bibitem{IEEEhowto:kopka}
%H.~Kopka and P.~W. Daly, \emph{A Guide to \LaTeX}, 3rd~ed.\hskip 1em plus
%  0.5em minus 0.4em\relax Harlow, England: Addison-Wesley, 1999.

%\end{thebibliography}

% biography section
% 
% If you have an EPS/PDF photo (graphicx package needed) extra braces are
% needed around the contents of the optional argument to biography to prevent
% the LaTeX parser from getting confused when it sees the complicated
% \includegraphics command within an optional argument. (You could create
% your own custom macro containing the \includegraphics command to make things
% simpler here.)
%\begin{biography}[{\includegraphics[width=1in,height=1.25in,clip,keepaspectratio]{mshell}}]{Michael Shell}
% or if you just want to reserve a space for a photo:

%\begin{IEEEbiography}{Michael Shell}
%Biography text here.
%\end{IEEEbiography}

% if you will not have a photo at all:
%\begin{IEEEbiographynophoto}{John Doe}
%Biography text here.
%\end{IEEEbiographynophoto}

% insert where needed to balance the two columns on the last page with
% biographies
%\newpage

%\begin{IEEEbiographynophoto}{Jane Doe}
%Biography text here.
%\end{IEEEbiographynophoto}

% You can push biographies down or up by placing
% a \vfill before or after them. The appropriate
% use of \vfill depends on what kind of text is
% on the last page and whether or not the columns
% are being equalized.

%\vfill

% Can be used to pull up biographies so that the bottom of the last one
% is flush with the other column.
%\enlargethispage{-5in}



% that's all folks
\end{document}


